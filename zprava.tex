\documentclass[12pt]{article}
\usepackage{epsf,epic,eepic,eepicemu}
%\documentstyle[epsf,epic,eepic,eepicemu]{article}
\usepackage[cp1250]{inputenc}


\begin{document}
%\oddsidemargin=-5mm \evensidemargin=-5mm \marginparwidth=.08in
%\marginparsep=.01in \marginparpush=5pt \topmargin=-15mm
%\headheight=12pt \headsep=25pt \footheight=12pt \footskip=30pt
%\textheight=25cm \textwidth=17cm \columnsep=2mm \columnseprule=1pt
%\parindent=15pt\parskip=2pt

\begin{center}
\bf Semestraln projekt MI-PAR 2010/2011:\\[5mm]
    Paraleln algoritmus pro een problmu\\[5mm]
       Jmno1 Pjmen1\\
       Jmno2 Pjmen2\\[2mm]
magistersk studijum, FIT VUT, Kolejn 550/2, 160 00 Praha 6\\[2mm]
\today
\end{center}

\section{Definice problmu a popis sekvennho algoritmu}

Popite problm, kter v program e. Jako vchoz pouijte text
zadn, kter rozite o pesn vymezen vech odchylek, kter jste
vi zadn bhem implementace provedli (nap.  pravy heuristick
funkce, organizace zsobnku, apod.). Zmite i ppadn i takov
prvky algoritmu, kter v zadn nebyly specifikovny, ale kter se
ukzaly jako dleit.  Dle popite vstupy a vstupy algoritmu
(formt vstupnch a vstupnch dat). Uvete tabulku nameench as
sekvennho algoritmu pro rzn velk data.

\section{Popis paralelnho algoritmu a jeho implementace v MPI}

Popite paraleln algoritmus, opt vyjdte ze zadn a pesn
vymezte odchylky, zvlt u algoritmu pro vyvaovn zte, hledn
drce, ci ukonen vpotu.  Popite a vysvtlete strukturu
celkovho paralelnho algoritmu na rovni proces v MPI a strukturu
kdu jednotlivch proces. Nap. jak je naimplementovna smyka pro
innost proces v aktivnm stavu i v stavu neinnosti. Jak jste
zvolili konstanty a parametry pro klovn algoritmu. Struktura a
smantika pkazov dky pro spoutn programu.

\section{Namen vsledky a vyhodnocen}

\begin{enumerate}
\item Zvolte ti instance problmu s takovou velikost vstupnch dat, pro kter m
sekvenn algoritmus asovou sloitost kolem 5, 10 a 15 minut. Pro
meen as potebn na ten dat z disku a uloen na disk
neuvaujte a zakomentujte ladc tisky, logy, zprvy a vstupy.
\item Mte paraleln as pi pouit $i=2,\cdot,32$ procesor na stch Ethernet a InfiniBand.
%\item Pri mereni kazde instance problemu na dany pocet procesoru spoctete pro vas algoritmus dynamicke delby prace celkovy pocet odeslanych zadosti o praci, prumer na 1 procesor a jejich uspesnost.
%\item Mereni pro dany pocet procesoru a instanci problemu provedte 3x a pouzijte prumerne hodnoty.
\item Z namench dat sestavte grafy zrychlen $S(n,p)$. Zjistte, zda a za jakych podmnek
dolo k superlinernmu zrychlen a pokuste se je zdvodnit.
\item Vyhodnote komunikan sloitost dynamickho vyvaovn zte a posute
vhodnost vmi implementovanho algoritmu pro hledn drce a dlen
zsobnku pri een vaeho problmu. Posute efektivnost a
klovatelnost algoritmu. Popite nedostatky va implementace a
navrhnte zlepen.
\item Empiricky stanovte
granularitu va implementace, tj., stupe paralelismu pro danou
velikost eenho problmu. Stanovte kritria pro stanoven mez, za
ktermi ji nen uinn rozkldat vpoet na men procesy, protoe
by komunikan nklady previly urychlen paralelnm vpotem.

\end{enumerate}

\section{Zvr}

Celkov zhodnocen semestrln prce a zkuenosti zskanch bhem
semestru.

\section{Literatura}

\appendix

\section{Nvod pro vkldn graf a obrzk do Latexu}

Nejjednodu zpsob vytvoen obrzku je pout vektorov grafick
editor (nap. xfig nebo jfig), ze kterho lze exportovat bu
\begin{itemize}
\item postscript formty (ps nebo eps formt) nebo
\item latex formty (v poad prost latex, latex s macry epic, eepic, eepicemu). Uveden poad odpovd rstu
komplikovanosti obrzk kter formt podporuje (prost latex macra
umonuj pouze jednoduch, epic makra nco mezi, je teba
vyzkouet).

\end{itemize}
Nsledujc pklady plat pro vechny ppady.

Obrzek v postscriptu, vycentrovan a na celou ku strnky, s
popisem a slem. Vimnete si, jak dit velikost obrazku.
\begin{figure}[ht]
\epsfysize=3cm \centerline{\epsfbox{VasObrazek.ps}} \caption{Popis
vaeho obrazku} \label{labelvasehoobrazku}
\end{figure}

Obrzek pouze vloen mezi dky textu, bez popisu a slovn.\\
\epsfxsize=1cm
\rule{0pt}{0pt}\hfill\epsfbox{VasObrazek.ps}\hfill\rule{0pt}{0pt}

Latexovsk obrzky maji ppony *.latex, *.epic, *.eepic, a
*.eepicemu, respective.
\begin{figure}[ht]
\begin{center}
\input VasObrazek.latex
\end{center}
\caption{Popis vaeho obrzku} \label{l1}
\end{figure}
Vyputenm zvorek {\tt figure} dostanete opt pouze rmeek v textu
bez sla a popisu.

Takhle jednodue mete poskldat obrzky vedle sebe.
\begin{center}
\setlength{\unitlength}{0.1mm}\input VasObrazek.epic
\hglue 5mm
\setlength{\unitlength}{0.15mm}\input VasObrazek.eepic
\hglue 5mm
\setlength{\unitlength}{0.2mm}\input VasObrazek.eepicemu
\end{center}
dit velikost latexovskych obrzk lze pkazem
\begin{verbatim}
\setlength{\unitlength}{0.1mm}
\end{verbatim}
kter mn mtko rastru obrzku, Tyto pkazy je ale souasn
nutn vyhodit ze souboru, kter xfig vygeneroval.

Pro vytven grafu lze pout program gnuplot, kter um generovat
postscriptovy soubor, ktery vlote do Latexu ve uvedenm
zpsobem.

\end{document}
